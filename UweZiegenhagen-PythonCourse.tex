%!TEX TS-program = pdflatex
\documentclass[ngerman]{beamer}

\usepackage[T1]{fontenc}
\usepackage{babel}
\usepackage{csquotes}
\setbeamertemplate{navigation symbols}{}

\setbeamercolor{block body alerted}{bg=alerted text.fg!10}
\setbeamercolor{block title alerted}{bg=alerted text.fg!20}
\setbeamercolor{block body}{bg=structure!10}
\setbeamercolor{block title}{bg=structure!20}
\setbeamercolor{block body example}{bg=green!10}
\setbeamercolor{block title example}{bg=green!20}

% https://tex.stackexchange.com/questions/339462/beamer-table-of-content-put-only-the-current-section-and-subsection-at-the-top
\AtBeginSection{%
    \begin{frame}
\begin{center}
        \LARGE\bfseries\tableofcontents[sections=\value{section}]
\end{center}
    \end{frame}
}

\usepackage{xcolor}
\usepackage{listings}
\lstset{upquote=true}



\lstdefinestyle{latex}{basicstyle=\ttfamily,morekeywords={usepackage,varobar,defbeamertemplate,definecolor,setbeamertemplate,usebeamertemplate,usetheme,gettwofromjobname,setbeamercolor,setbeamerfont,tiny,addtobeamertemplate,itshape,sffamily,twemoji}}

\lstdefinestyle{latexf}{style=latex, language=[LaTeX]TeX,
basicstyle=\ttfamily\footnotesize}

\lstdefinestyle{Python}{language=Python,basicstyle=\ttfamily\footnotesize,morekeywords={read_csv}}

\usepackage{attachfile}
\newcommand{\ta}[1]{\textattachfile[color=1 0 0]{#1}{Code}}

\newcommand{\pypy}[2]{\lstinputlisting[language={Python},caption={#1 \ta{#2}}]{#2}}

\makeatletter
\lstdefinestyle{ausgabe}{
  basicstyle=\ttfamily\scriptsize,%
  backgroundcolor=\color{lightgray}%
}
\makeatother

\lstnewenvironment{ausgabe}{\lstset{style=ausgabe}}{} 

\definecolor{colBack}{rgb}{1,1,0.9}
\definecolor{colKeys}{rgb}{0,0,1}
\definecolor{colIdentifier}{rgb}{0,0,0}
\definecolor{colComments}{rgb}{1,0,0}
\definecolor{colString}{rgb}{0,0.5,0}

\lstset{literate=%
    {Ö}{{\"O}}1
    {Ä}{{\"A}}1
    {Ü}{{\"U}}1
    {ß}{{\ss}}1
    {ü}{{\"u}}1
    {ä}{{\"a}}1
    {ö}{{\"o}}1
    {~}{{\textasciitilde}}1
}


\lstset{%
    float=hbp,%
    basicstyle=\ttfamily\footnotesize, %
    identifierstyle=\color{colIdentifier}, %
    keywordstyle=\color{colKeys}, %
    stringstyle=\color{colString}, %
    commentstyle=\color{colComments}, %
    columns=flexible, %
    tabsize=2, %
    frame=single, %
    extendedchars=true, %
    showspaces=false, %
    showstringspaces=false, %
    numbers=left, %
    numberstyle=\tiny, %
    breaklines=true, %
    backgroundcolor=\color{colBack}, %
    breakautoindent=true, %
    captionpos=b,%
    language={Python},
    morekeywords={copyfile,write,unlink}
}


\usepackage[T1]{fontenc}

\newcommand{\bild}[1]{\fbox{\includegraphics[width=0.2\textwidth]{Presentation-AnnArbor-#1}}}

\author{Uwe Ziegenhagen}
\title{Python \& pandas}
\subtitle{A one day course}
\institute{\url{github.com/UweZiegenhagen/OneDayPythonPandasCourse}}
\date{Cologne, \today}


\begin{document}

\begin{frame}

\maketitle

\end{frame}

\section{Introduction}

\begin{frame}[fragile]
\frametitle{Why Python/pandas?}

\begin{itemize}
	\item You \textit{have} a CSV-file with semicolon as column separator and comma as decimal separator
	\item You \textit{need} a CSV-file with comma as column separator and dot as decimal separator
\end{itemize}

\begin{lstlisting}
import pandas as pd

df = pd.read_csv('myfile.csv', sep=';', decimal = ',')
df = pd.to_csv('myfile.csv', sep=';', decimal = ',')
\end{lstlisting}


\end{frame}

\begin{frame}
\frametitle{Limits of this Course}

\begin{itemize}
\item Basis for this course is a semester-long course I held at the FOM (\enquote{Fachhochschule für Ökonomie und Management}) in Cologne 
\item It is not a \textit{full} course, we would need a whole week for this.
\item We will skip many interesting things (that you do not necessarily need for your job)
\item Goal: Teach you enough Python to a) read and b) understand Python-Code and c) write smaller programs relevant for your job 
\end{itemize}
\end{frame}



\begin{frame}

\tableofcontents

\end{frame}


\begin{frame}
\frametitle{Python}

\begin{itemize}
	\item Invented by Guido van Rossum at the \enquote{Centrum Wiskunde \& Informatica} in Amsterdam as successor for the teaching language ABC
	\item Current version is 3.11
	\item For a long time, Python 3.x and Python 2.7 existed together
	\item Python 2.7 support expired in 2019:
	\item How to spot 2.7 code: $\rightarrow$ \texttt{print 'hello'} instead of \texttt{print('hello')}
\end{itemize}

\end{frame}

\begin{frame}
\frametitle{Python versus Java \& C}

Python code is often much slower than C or Java but\ldots

\begin{itemize}
\item the implementation time for Python is way faster
\item speeds only matters sometimes, not always
\item many computing-intensive Python modules use\newline C/C++ modules \enquote{under the hood}
\item bad C-Code is slower than good Python-code
\end{itemize}
\end{frame}


\begin{frame}
\frametitle{pandas}

\begin{itemize}
	\item A Python library for data wrangling and management
	\item Invented by Wes McKinney during his time at AQR Capital Management
	\item In his own words:
\enquote{I tell them that it enables people to analyze and work with data who are not expert computer scientists,} he says. \enquote{You still have to write code, but it’s making the code intuitive and accessible. It helps people move beyond just using Excel for data analysis.}\footnote{\url{qz.com/1126615/the-story-of-the-most-important-tool-in-data-science/}}
\end{itemize}

\end{frame}

\section{Python}

\begin{frame}
\frametitle{Spyder}

\begin{itemize}
	\item We will use the Spyder5 IDE with Python 3.9.9, make sure it is installed
\end{itemize}

\begin{center}
\includegraphics[width=\textwidth]{Pictures/Spyder5}
\end{center}

\end{frame}

\begin{frame}[fragile]
\frametitle{Python as a Calculator}

\begin{itemize}
\item Spyder5 runs an IPython kernel, this runs our programs
\item We can also use it as a calculator
\end{itemize}

\begin{lstlisting}[style=Python]
In [1]: 4*5.4
Out[1]: 21.6

In [2]: 4/12
Out[2]: 0.3333333333333333

In [3]: _*3
Out[3]: 1.0

'hello'
Out[4]: 'hello'
\end{lstlisting}

\end{frame}


\begin{frame}[fragile]
\frametitle{Python as a Calculator}

\begin{lstlisting}[style=Python]

In [1]: 4%2
Out[1]: 0

In [2]: 5%2 # Modulo
Out[2]: 1

In [3]: 3**3
Out[3]: 27

In [4]: 5//2
Out[4]: 2
\end{lstlisting}

\end{frame}

\begin{frame}
\frametitle{Priority of Operators}

\begin{itemize}
\item Round brackets have highest priority
\item followed by Power
\item followed by multiplication and division
\item followed by addition and substraction
\end{itemize}

\begin{block}{TODO4U:}
$\Rightarrow$ Solve exercise sheet 1!
\end{block}

\end{frame}

\begin{frame}[fragile]
\frametitle{Basic Input \& Output}

\lstinputlisting[style=Python]{Codes/Basic_IO.py}

\begin{itemize}
	\item \lstinline[style=Python]{input()} only reads strings	
	\item If you need a number, you need to convert it
	\item there are better ways than \lstinline[style=Python]{print()} for logging, but it works\ldots 
	\item f-Strings (last row) are recommended for mixed output!
	\end{itemize}
\end{frame}

\begin{frame}
\frametitle{Rules for Variables}

\begin{itemize}
\item must start with a letter or \textunderscore
\item Case-sensitivity: 'A' is not 'a'
\item Recommendation: small letters
\item Let them speak for themselves: 'diameter' is good, 'd' is bad 
\end{itemize}
\end{frame}


\begin{frame}
\frametitle{Reserved Keywords}

The following  keywords are reserviert and must not be used for variables' names.

\begin{center}
\begin{tabular}{p{0.15\textwidth}p{0.15\textwidth}p{0.15\textwidth}p{0.15\textwidth}p{0.15\textwidth}}
and	&	as	&	assert	&	break	&	class	\\
continue	&	def	&	del	&	elif	&	else	\\
except	&	False	&	finally	&	for	&	from	\\
global	&	if	&	import	&	in	&	is	\\
lambda	&	None	&	nonlocal	&	not	&	or	\\
pass	&	raise	&	return	&	True	&	try	\\
while	&	with	&	yield	&		&		\\
\end{tabular}
\end{center}
\end{frame}



\subsection{Datatypes}

\begin{frame}
\frametitle{Datentypen}


\begin{itemize}
\item Integer (integer numbers)
\item Float (Floating point number)
\item Strings
\item Booleans
\item Complex numbers
\end{itemize}
\end{frame}

\begin{frame}
\frametitle{Integer}

\begin{itemize}
\item unlimited length
\item must not start with 0 if they shall represent decimal numbers 
\item Leading 0 by representation in hex-, binary- or Octal system:
\begin{description}
\item[\texttt{0b/0B}] binary
\item[\texttt{0x/0X}] hex
\item[\texttt{0o/0O}] octal
\end{description}
\item Functions \texttt{hex()}, \texttt{bin()}, \texttt{oct()} for conversions into string, are internally represented as decimal numbers 
\end{itemize}
\end{frame}

\begin{frame}
\frametitle{Float}

\begin{itemize}
	\item floating point numbers
	\item \texttt{3.1415927}
	\item \texttt{3.1e8}
	\item Hint: Not every floating point number can be represented \textit{exactly} (\enquote{Floating-Point Arithmetic}) 
	\item \url{docs.python.org/3/tutorial/floatingpoint.html}
\end{itemize}
\end{frame}

\begin{frame}[containsverbatim]
\frametitle{Strings}

\begin{itemize}
\item Single or double quotes
\item For multiline strings:

\begin{itemize}
	\item Triple double or single quotes
	\item Alternatively backslash at the end of the line 
\end{itemize}

\item Python has numerous functions for strings, more on this later
\item Comments start with a hash \#
\end{itemize}


\begin{lstlisting}[style=Python]
a = "I am a string"

b = 'me too'

c = """I am 
as string as well """

# I am a comment
\end{lstlisting}


\end{frame}

\begin{frame}[containsverbatim]
\frametitle{Booleans}

\begin{itemize}
	\item Named after George Boole
	\item 1854: \enquote{An investigation into the Laws of Thought}
	\item Essence of modern computing
	\item Boolean operators or ($\cup$ ),  and ($\cap$ ), \texttt{not}
\end{itemize}

\begin{lstlisting}[language={Python}]
a = True
b = False 

a == b #False
a or b # True
a and b # False
a and not b # True
not a and b # False
\end{lstlisting}
\end{frame}



\begin{frame}[containsverbatim]
\frametitle{Type Conversions}

\begin{itemize}
\item Mixing strings and floats/integers requires explicit type conversion using the \lstinline[language={Python}]{str()} function
\item Hint: pandas provides \lstinline[morekeywords={astype},language={Python}]{.astype(<Datatype>)} 
\end{itemize}

\begin{lstlisting}[language={Python}]
>>> a + str(b)
'abc123'
>>> a+str(c)
'abc3.141'
>>> a*str(b)
Traceback (most recent call last):
  File "<stdin>", line 1, in <module>
TypeError: can't multiply sequence by non-int of type 'str'

>>> 'a' * 3
'aaa'

>>>3 * 'a'
'aaa'
\end{lstlisting}

%\begin{center}
%\fomh{$\Rightarrow$ Bitte Arbeitsblatt 2 bearbeiten!}
%\end{center}

\end{frame}

\subsection{Functions}


\begin{frame}
\frametitle{Functions}

\begin{itemize}
%\item Bereits bekannte Funktionen:  \lstinline[language={Python}]{id()} , \lstinline[language={Python}]{len()}  und \lstinline[language={Python}]{type()} 
\item Functions: named sequence of commands
\item Purpose: encapsule code for multiple calls
\item Can take arguments as input, can have return values
\item Define a new function as
\begin{itemize}
	\item Determine name
	\item Determine arguments
	\item Define function code
	\end{itemize}
\end{itemize}
\end{frame}


\begin{frame}
\frametitle{Built-in Functions}

\begin{tabular}{lllll}
abs	&	delattr	&	hash	&	memoryview	&	set	\\
all	&	dict	&	help	&	min	&	setattr	\\
any	&	dir	&	hex	&	next	&	slice	\\
ascii	&	divmod	&	id	&	object	&	sorted	\\
bin	&	enumerate	&	input	&	oct	&	staticmethod	\\
bool	&	eval	&	int	&	open	&	str	\\
breakpoint	&	exec	&	isinstance	&	ord	&	sum	\\
bytearray	&	filter	&	issubclass	&	pow	&	super	\\
bytes	&	float	&	iter	&	print	&	tuple	\\
callable	&	format	&	len	&	property	&	type	\\
chr	&	frozenset	&	list	&	range	&	vars	\\
classmethod	&	getattr	&	locals	&	repr	&	zip	\\
compile	&	globals	&	map	&	reversed	&	\textunderscore \textunderscore import \textunderscore \textunderscore	\\
complex	&	hasattr	&	max	&	round	&		\\
\end{tabular}
\end{frame}



\begin{frame}[containsverbatim]
\frametitle{Functions in C and Python}

\lstinputlisting[language={C},caption={addTwoNumbers.c \ta{Codes/addTwoNumbers.c}}]{Codes/addTwoNumbers.c}\vspace*{-0.5em}

\lstinputlisting[language={Python},caption={addTwoNumbers.py \ta{Codes/addTwoNumbers.py}}]{Codes/addTwoNumbers.py}

\end{frame}

\begin{frame}[containsverbatim]
\frametitle{Simple Functions}
\framesubtitle{Empty Functions}

\begin{itemize}
\item \lstinline[language={Python}]{pass} is often used in the development process
\item Useful, when parts of the code are not ready, yet
\item without \lstinline[language={Python}]{pass} one gets \texttt{IndentationError: expected an indented block} error
\end{itemize}

\lstinputlisting[language={Python},caption={funktion-12.py \ta{Codes/funktion-12.py}}]{Codes/funktion-12.py}

\end{frame}


\begin{frame}[containsverbatim]
\frametitle{Simple Functions}

\begin{itemize}
	\item No parameter
	\item No return value (\texttt{void})
\end{itemize}

\lstinputlisting[language={Python},caption={funktion-01.py \ta{Codes/funktion-01.py}}]{Codes/funktion-01.py}

\begin{ausgabe}
Hello
\end{ausgabe}

\end{frame}



\begin{frame}[containsverbatim]
\frametitle{Functions with Arguments}

\begin{itemize}
	\item Function with one argument \texttt{text} 
	\item Error message, when argument is missing
\end{itemize}

\lstinputlisting[language={Python},caption={funktion-02.py \ta{Codes/funktion-02.py}}]{Codes/funktion-02.py}

\begin{ausgabe}
Hello World!
\end{ausgabe}


\end{frame}

\begin{frame}[containsverbatim]
\frametitle{Functions with Arguments}

\begin{itemize}
	\item Two arguments, \texttt{text} and \texttt{count}
\end{itemize}

\lstinputlisting[language={Python},caption={funktion-03.py \ta{Codes/funktion-03.py}}]{Codes/funktion-03.py}

\begin{ausgabe}

Hello!
Hello!Hello!Hello!

\end{ausgabe}

\end{frame}

\begin{frame}[containsverbatim]
\frametitle{Functions with Arguments}

\begin{itemize}
	\item Setting standard values for parameters
	\item Allows calling the function without parameters
\end{itemize}

\lstinputlisting[language={Python},caption={funktion-04.py \ta{Codes/funktion-04.py}}]{Codes/funktion-04.py}

\begin{ausgabe}
Hello CGNHello CGN
Hello!Hello!
Hello!Hello!Hello!
\end{ausgabe}

\end{frame}

\begin{frame}[containsverbatim]
\frametitle{Functiona with return values}

\begin{itemize}
	\item Functions can return values for further processing
\end{itemize}

\lstinputlisting[language={Python},caption={funktion-05.py \ta{Codes/funktion-05.py}}]{Codes/funktion-05.py}

\begin{ausgabe}
HuhuHuhu
\end{ausgabe}

\end{frame}


\begin{frame}[containsverbatim]
\frametitle{Functions with multiple return values}

\begin{itemize}
	\item Functions can have more than one return value
	\item Function then return tuples, an unmutuable list of values\footnote{more on this later}
	\item Dealing with the tuple is called \enquote{Unpacking}
	\item Remark: Parameter \texttt{sep} in the example is a parameter \newline of the \lstinline[language={Python}]{print()}  function
\end{itemize}

\lstinputlisting[language={Python},caption={funktion-06.py \ta{Codes/funktion-06.py}}]{Codes/funktion-06.py}

\vspace*{-1em}
\begin{ausgabe}
2>HuhuHuhu
\end{ausgabe}
\end{frame}

\begin{frame}
\frametitle{Functions with variable count of arguments}

\begin{itemize}
\item Parameter with *: variable count of arguments
\item Parameter with **: variable count of key-value arguments
\end{itemize}

\lstinputlisting[language={Python},caption={funktion-08.py \ta{Codes/funktion-08.py}}]{Codes/funktion-08.py}

\lstinputlisting[language={Python},caption={funktion-09.py \ta{Codes/funktion-09.py}}]{Codes/funktion-09.py}

\end{frame}


\subsection{Flow control}

\begin{frame}
\frametitle{Flow control}

\begin{itemize}
\item Input \& Output
\begin{itemize}
	\item input
	\item print
\end{itemize}

\item Branching
\begin{itemize}
	\item if else elif
\end{itemize}

\item Loops
\begin{itemize}
	\item for
	\item while
\end{itemize}

\end{itemize}
\end{frame}

\begin{frame}[containsverbatim]
\frametitle{Branching}

\begin{itemize}
\item \lstinline[language={Python}]{if <condition>:}
\end{itemize}

\lstinputlisting[language={Python},caption={if-01.py \ta{Codes/if-01.py}} ]{Codes/if-01.py}

\begin{ausgabe}
Longer than 8 characters!
\end{ausgabe}
\end{frame}

\begin{frame}[containsverbatim]
\frametitle{Branching}

\begin{itemize}
\item There is no \texttt{switch()} in Python
\item \lstinline[language={Python}]{if} statements can be used multiple times
\item maybe not the most pythonic approach
\end{itemize}

\lstinputlisting[language={Python},caption={if-02.py \ta{Codes/if-02.py}}]{Codes/if-02.py}

\begin{ausgabe}
Köln: Shorter or equal to 8 characters!
Düsseldorf: Longer than 8 characters!
\end{ausgabe}
\end{frame}


\begin{frame}[containsverbatim]
\frametitle{Branching}

\begin{itemize}
\item more \enquote{pythonic}:  \lstinline[language={Python}]{else:}
\end{itemize}

\lstinputlisting[language={Python},caption={if-03.py \ta{Codes/if-03.py}}]{Codes/if-03.py}

\begin{ausgabe}
Köln: Shorter or equal to 8 characters!
Düsseldorf: Longer than 8 characters!
\end{ausgabe}

\end{frame}

\begin{frame}[containsverbatim]
\frametitle{Branching}

\begin{itemize}
\item Nesting von \lstinline[language={Python}]{if <condition>: else:} leads to confusing code
\end{itemize}

\lstinputlisting[language={Python},caption={if-04.py \ta{Codes/if-04.py}}]{Codes/if-04.py}

\begin{ausgabe}
Köln: Shorter or equal 5 chars!
Berlin: Longer than 5, shorter than 8
Düsseldorf: Longer than 8 chars!
\end{ausgabe}

\end{frame}

\begin{frame}[containsverbatim]
\frametitle{Branching}

\begin{itemize}
\item \lstinline[language={Python}]{if <condition>: else:} can be shortened to \lstinline[language={Python}]{elif:} 
\end{itemize}

\lstinputlisting[language={Python},caption={if-05.py \ta{Codes/if-05.py}}]{Codes/if-05.py}

\begin{ausgabe}
Köln: Shorter or equal 5 chars!
Berlin: Longer than 5, shorter than 8
Düsseldorf: Longer than 8 chars!
\end{ausgabe}


\end{frame}


\begin{frame}[containsverbatim]
\frametitle{Loops}
\framesubtitle{\lstinline[language={Python}]{for}}

\begin{itemize}
\item \lstinline[language={Python}]{for} loops iterate over a sequence
\item sequence can be a string, a liste, etc. 
\item \lstinline[language={Python}]{range(start, end, stepsize=1)} creates numerical sequence from \texttt{start} until below (!) \texttt{end} with step size \texttt{stepsize}
\end{itemize}

\lstinputlisting[language={Python},caption={for-01.py \ta{Codes/for-01.py}}]{Codes/for-01.py}

\end{frame}


\begin{frame}[containsverbatim]
\frametitle{Loops}
\framesubtitle{\lstinline[language={Python}]{while}}

\begin{itemize}
\item \lstinline[language={Python}]{while} loop runs, until condition is not fulfilled anymore
\end{itemize}

\lstinputlisting[language={Python},caption={while-01.py \ta{Codes/while-01.py}}]{Codes/while-01.py}

\end{frame}

\begin{frame}[containsverbatim]
\frametitle{Loops}
\framesubtitle{\lstinline[language={Python}]{break} and \lstinline[language={Python}]{continue}}

\begin{itemize}
\item \lstinline[language={Python}]{break} and \lstinline[language={Python}]{continue} influence loops
\item \lstinline[language={Python}]{break} can e.\,g. be used to exit a loop 
\end{itemize}

\lstinputlisting[language={Python},caption={while-02.py \ta{Codes/while-02.py}}]{Codes/while-02.py}

\end{frame}




\section{pandas}


\section{Links}


\begin{frame}
\frametitle{Links}

\begin{itemize}
	\item \url{https://pandas.pydata.org/pandas-docs/stable/user_guide/10min.html}
	\item \url{https://pandas.pydata.org}
	\item 
	\item 
	\item 
	\item 
\end{itemize}



\end{frame}


\end{document}


\lstinline[style=Python]{input()}



\begin{frame}
\frametitle{Built-in Functions I}

* means: \enquote{no relevance here}

\begin{description}
\item[abs()] absolute value of a number
\item[all()] checks if all elements of an iterable are true *
\item[any()] checks if at least any element of an iterable is true *
\item[ascii()] ASCII representation of an object *
\item[bin()] Conversion to binary
\item[bool()] returns boolean value for expression *
\end{description}
\end{frame}

\begin{frame}
\frametitle{Built-in Functions II}

\begin{description}
\item[breakpoint()] im Debugging genutzt*
\item[bytearray()] erzeugt Array of Bytes*
\item[bytes()] erzeugt neues Byte-Objekt*
\item[callable()] Prüfung, ob Objekt aufrufbar ist*
\item[chr()] Unicode-String-Repräsentation einer Zahl
\item[classmethod()] wandelt Methode in Klassenmethode um*
\end{description}
\end{frame}


\begin{frame}
\frametitle{Built-in Functions III}

\begin{description}
\item[compile()] übersetzt String in exec-baren Code*
\item[complex()] erzeugt komplexe Zahl*
\item[delattr()] löscht Attribut aus Objekt
\item[dict()] erstellt ein dictionary Objekt
\item[dir()] Variablenliste im aktuellen Scope
\item[divmod()] gibt Tupel aus Integer Division und Rest zurück
\end{description}
\end{frame}


\begin{frame}
\frametitle{Built-in Functions III}

\begin{description}
\item[enumerate()] erzeugte Liste von Zahl, Item aus Iterables
\item[eval()] parst und führt String-Ausdruck aus
\item[exec()] führt dynamisch erzeugten Code aus
\item[filter()] genutzt in funktionaler Programmierung
\item[float()] erstellt float aus Zahl oder String
\item[format()] erstellt formatierten String
\end{description}
\end{frame}


\begin{frame}
\frametitle{Built-in Functions IV}

\begin{description}
\item[frozenset()] erzeugt unmutuable Set*
\item[getattr()] gibt Wert von Attribut eines Objekts zurück
\item[globals()] gibt dict der globalen Symboltabelle zurück
\item[hasattr()] prüft ob Objekt ein Attribut hat
\item[hash()] gibt integer Hash-Wert für Objekt zurück
\item[help()] ruft die Hilfe auf
\end{description}
\end{frame}


\begin{frame}
\frametitle{Built-in Functions V}

\begin{description}
\item[hex()] Hexadezimaldarstellung
\item[id()] Gibt die interne ID eines Objekts zurück
\item[input()] liest von der Tastatur einen String
\item[int()] erzeugt integer aus Zahl oder String
\item[isinstance()] prüft, ob Objekt von Typ x ist
\item[issubclass()] prüft, ob Klasse Subklasse von x ist
\end{description}
\end{frame}


\begin{frame}
\frametitle{Built-in Functions VI}

\begin{description}
\item[iter()] iteriert über Sequenz
\item[len()] Länge eines Objekt
\item[list()] erzeugt Liste
\item[locals()] erzeugt Update für die 
\item[map()] genutzt in funktionaler Programmierung
\item[max()] gibt Maximum zurück
\end{description}
\end{frame}


\begin{frame}
\frametitle{Built-in Functions VII}

\begin{description}
\item[memoryview()] gibt memoryview eines Objekts wieder*
\item[min()] gibt Minimum zurück
\item[next()] gibt nächstes Objekt von Iterator zurück
\item[object()] gibt neues Objekt zurück*
\item[oct()] Oktaldarstellung einer Zahl*
\item[open()] öffnet Datei zum Lesen/Schreiben
\end{description}
\end{frame}


\begin{frame}
\frametitle{Built-in Functions VIII}

\begin{description}
\item[ord()] Inverse von chr(), gibt Zahl für Zeichen aus
\item[pow()] analog zu x**y
\item[print()] gibt Objekte aus auf Kommandozeile, in Datei
\item[property()] gibt Property-Attribut aus*
\item[range()] erzeugt eine Zahlenliste
\item[repr()] gibt ASCII-Repräsentation eines Objekts wieder*
\end{description}
\end{frame}


\begin{frame}
\frametitle{Built-in Functions IX}

\begin{description}
\item[reversed()] gibt umgekehrte Sequence zurück
\item[round()] rundet Zahl
\item[set()] erstellt Menge
\item[setattr()] Gegenstück zu getattr()
\item[slice()] gibt slice-Objekt zurück*
\item[sorted()] gibt sortierte Liste zurück
\end{description}
\end{frame}


\begin{frame}
\frametitle{Built-in Functions X}

\begin{description}
\item[staticmethod()] wandelt Methode in statische Methode um*
\item[str()] erzeugt String aus Objekt
\item[sum()] berechnet Summe aus 
\item[super()] gibt Proxy-Objekt zurück*
\item[tuple()] erzeugt Tupel aus Iterable
\item[type()] gibt Typ eines Objekts wieder
\end{description}
\end{frame}


\begin{frame}
\frametitle{Built-in Functions XI}

\begin{description}
\item[vars()] gibt dict-Attribut eines Objekts wider
\item[zip()] aggregiert iterables in einen Iterator
\item[\textunderscore \textunderscore import()\textunderscore \textunderscore] Funktion zum Anpassen von import-Statements*
\end{description}



\end{frame}
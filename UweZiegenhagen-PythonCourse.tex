%!TEX TS-program = pdflatex
\documentclass[ngerman]{beamer}

\usepackage[T1]{fontenc}
\usepackage{babel}
\usepackage{csquotes}
\setbeamertemplate{navigation symbols}{}


% https://tex.stackexchange.com/questions/339462/beamer-table-of-content-put-only-the-current-section-and-subsection-at-the-top
\AtBeginSection{%
    \begin{frame}
\begin{center}
        \LARGE\bfseries\tableofcontents[sections=\value{section}]
\end{center}
    \end{frame}
}

\usepackage{xcolor}
\usepackage{listings}
\lstset{upquote=true}



\lstdefinestyle{latex}{basicstyle=\ttfamily,morekeywords={usepackage,varobar,defbeamertemplate,definecolor,setbeamertemplate,usebeamertemplate,usetheme,gettwofromjobname,setbeamercolor,setbeamerfont,tiny,addtobeamertemplate,itshape,sffamily,twemoji}}

\lstdefinestyle{latexf}{style=latex, language=[LaTeX]TeX,
basicstyle=\ttfamily\footnotesize}


\definecolor{colBack}{rgb}{1,1,0.9}
\definecolor{colKeys}{rgb}{0,0,1}
\definecolor{colIdentifier}{rgb}{0,0,0}
\definecolor{colComments}{rgb}{1,0,0}
\definecolor{colString}{rgb}{0,0.5,0}

\lstset{literate=%
    {Ö}{{\"O}}1
    {Ä}{{\"A}}1
    {Ü}{{\"U}}1
    {ß}{{\ss}}1
    {ü}{{\"u}}1
    {ä}{{\"a}}1
    {ö}{{\"o}}1
    {~}{{\textasciitilde}}1
}


\lstset{%
    float=hbp,%
    basicstyle=\ttfamily\footnotesize, %
    identifierstyle=\color{colIdentifier}, %
    keywordstyle=\color{colKeys}, %
    stringstyle=\color{colString}, %
    commentstyle=\color{colComments}, %
    columns=flexible, %
    tabsize=2, %
    frame=single, %
    extendedchars=true, %
    showspaces=false, %
    showstringspaces=false, %
    numbers=left, %
    numberstyle=\tiny, %
    breaklines=true, %
    backgroundcolor=\color{colBack}, %
    breakautoindent=true, %
    captionpos=b,%
    language={Python},
    morekeywords={copyfile,write,unlink}
}


\usepackage[T1]{fontenc}

\newcommand{\bild}[1]{\fbox{\includegraphics[width=0.2\textwidth]{Presentation-AnnArbor-#1}}}

\author{Uwe Ziegenhagen}
\title{Python \& pandas}
\subtitle{A one day course}
\institute{\url{github.com/UweZiegenhagen/OneDayPythonPandasCourse}}
\date{Cologne, \today}


\begin{document}

\begin{frame}

\maketitle

\end{frame}

\section{Introduction}

\begin{frame}[fragile]
\frametitle{Why Python/pandas?}

\begin{itemize}
	\item You \textit{have} a CSV-file with semicolon as column separator and comma as decimal separator
	\item You \textit{need} a CSV-file with comma as column separator and dot as decimal separator
\end{itemize}

\begin{lstlisting}
import pandas as pd

df = pd.read_csv('myfile.csv', sep=';', decimal = ',')
df = pd.to_csv('myfile.csv', sep=';', decimal = ',')
\end{lstlisting}


\end{frame}

\begin{frame}

\tableofcontents

\end{frame}


\begin{frame}
\frametitle{Python}

\begin{itemize}
	\item Invented by Guido van Rossum at the \enquote{Centrum Wiskunde \& Informatica} in Amsterdam as successor for the teaching language ABC
	\item Current version is 3.11
	\item For a long time, Python 3.x and Python 2.7 existed together
	\item Python 2.7 support expired in 2019:
	\item How to spot 2.7 code $\rightarrow$ \texttt{print 'hello'} instead of \texttt{print('hello')}
\end{itemize}

\end{frame}

\begin{frame}
\frametitle{pandas}

\begin{itemize}
	\item A Python library for data wrangling and management
	\item Invented by Wes McKinney during his time at AQR Capital Management
	\item In his own words:
\enquote{I tell them that it enables people to analyze and work with data who are not expert computer scientists,} he says. \enquote{You still have to write code, but it’s making the code intuitive and accessible. It helps people move beyond just using Excel for data analysis.}\footnote{\url{qz.com/1126615/the-story-of-the-most-important-tool-in-data-science/}}
\end{itemize}

\end{frame}

\section{Python}

\begin{frame}
\frametitle{}


\end{frame}

\section{pandas}


\section{Links}


\begin{frame}
\frametitle{Links}

\begin{itemize}
	\item \url{https://pandas.pydata.org/pandas-docs/stable/user_guide/10min.html}
	\item \url{https://pandas.pydata.org}
	\item 
	\item 
	\item 
	\item 
\end{itemize}



\end{frame}




\end{document}
